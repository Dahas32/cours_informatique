\documentclass{article}
\usepackage[french]{babel}
\usepackage[T1]{fontenc}
\usepackage[utf8]{inputenc}
\usepackage{graphicx} % Required for inserting images
\usepackage{listingsutf8}
\lstset{%
    inputencoding=utf8,
    literate=
        {é}{{\'e}}{1}%
        {è}{{\`e}}{1}%
        {à}{{\`a}}{1}%
        {â}{{\^a}}{1}%
        {ç}{{\c{c}}}{1}%
        {œ}{{\oe}}{1}%
        {ù}{{\`u}}{1}%
        {É}{{\'E}}{1}%
        {È}{{\`E}}{1}%
        {À}{{\`A}}{1}%
        {Ç}{{\c{C}}}{1}%
        {Œ}{{\OE}}{1}%
        {Ê}{{\^E}}{1}%
        {ê}{{\^e}}{1}%
        {î}{{\^i}}{1}%
        {ï}{{\"i}}{1}%
        {ô}{{\^o}}{1}%
        {û}{{\^u}}{1}%
}
\usepackage{fullpage}
\usepackage{setspace}
\usepackage{todonotes}
\usepackage{amsthm}
\usepackage[french, frenchkw, boxed, linesnumbered]{algorithm2e}

\newtheoremstyle{exostyle}% style name
{10pt}% above space
{10pt}% below space
{}% body font
{}% indent amount
{\scshape\bfseries\large}% head font
{\hfill\vspace{5pt}\newline}% post head punctuation
{0pt}% Space after theorem head
{\hfill\thmname{#1}\thmnumber{ #2} -- \thmnote{ #3}}% head spec

\newtheoremstyle{partiestyle}% style name
{1em}% above space
{1em}% below space
{}% body font
{}% indent amount
{\bfseries}% head font
{\vspace{.5em}\newline}% post head punctuation
{0em}% Space after theorem head
{\thmnumber{#2} \thmnote{ #3}}% head spec

\newtheoremstyle{questionstyle}% style name
{.5em}% above space
{.5em}% below space
{}% body font
{}% indent amount
{\bfseries}% head font
{}% post head punctuation
{0em}% Space after theorem head
{Question \thmnumber{#2 }}% head spec

\theoremstyle{exostyle}
\newtheorem{exo}{Exercice}

\theoremstyle{partiestyle}
\newtheorem{partie}{}[exo]

\theoremstyle{questionstyle}
\newtheorem{question}{Question}[exo]
\newtheorem{questionpartie}{Question}[partie]

\title{Devoir Surveillé 1 Programmation et Algorithmes}
\author{L1 MPCI}
\date{23 février 2024 - Durée: 3h}

\begin{document}

\maketitle

\begin{center}
{\em\bf Lorsque l'on vous demande d'écrire de décrire ou de donner un algorithme cela signifiera toujours en donner un pseudo-code, justifier de son exactitude et de sa complexité}

~\\

{\em On rappelle qu'aucun document ni équipement électrique ou électronique n'est autorisé. }
\end{center}

\vspace*{1cm}
Le but de ce devoir est de résoudre de quatre façons différentes {\bf le problème de trouver l'élément majoritaire}.

\paragraph*{Définition.}\'Etant donné un tableau $T$, un {\it\bf élément majoritaire (de $T$)} est une valeur $x$ telle que strictement plus de la moitié des éléments de $T$ sont égaux à $x$.

\paragraph*{}Par exemple :
\begin{itemize}
    \item le tableau $[2,4,5,4,5,4, 5, 4, 4]$ admet $4$ comme élément majoritaire,
    \item le tableau $[2,2,3,6,4,3,2,2,3,3,2,2]$ n'a pas d'élément majoritaire. 
\end{itemize}
    

\paragraph*{Les exercices :}
\begin{itemize}
\item sont au nombre de 4;
\item sont indépendants;
\item ont un début plus facile que la fin.
\end{itemize}

\vspace*{2cm}
\begin{center}
    {\em\bf\sc\Large Pour préserver la santé mentale du correcteur rendez une copie par exercice.}
    
\end{center}

\clearpage

\begin{exo}[Encadrement du problème]

\begin{partie}[Préliminaire]
Combien d'éléments majoritaires peut avoir un tableau ? Prouvez-le rigoureusement.
\end{partie}
\begin{partie}[Borne minimum]
\begin{questionpartie}
    Montrez que quelle que soit la taille $n > 2$ du tableau et quel que soit $0 \leq i < n$, il existe un tableau $T$ d'élément majoritaire $e$ tel que $T[i] \neq e$.
\end{questionpartie}
\begin{questionpartie}
    Montrez que quelle que soit la taille $n \geq 1$ du tableau et quel que soit $0 \leq i < n$, il existe un tableau $T$ d'élément majoritaire $e$ tel que $T[i] = e$.
\end{questionpartie}
\begin{questionpartie}
    Montrez que quels que soient $n>2$ et $0 \leq i < n$, il existe toujours deux tableaux d'entiers $T$ et $T'$ de taille $n$ tels que :
    \begin{itemize}
        \item $T[j] = T'[j]$ pour tout $j \neq i$,
        \item $T$ admet un élément majoritaire,
        \item $T'$ n'admet pas d'élément majoritaire.
    \end{itemize}
\end{questionpartie}
\begin{questionpartie}
    Déduisez-en que la complexité du problème de l'élément majoritaire est en $\Omega(n)$, avec $n$ la taille du tableau (il n'existe pas d'algorithme permettant de résoudre le problème de l'élément majoritaire en strictement moins de $n$ opérations pour tout $n > N_0$).
\end{questionpartie}
\end{partie}
\begin{partie}[Borne maximum]

\begin{questionpartie}
    Donnez le pseudo-code et la preuve d'une fonction $\textsc{compte}$ de complexité $\mathcal{O}(n)$ telle que $\textsc{compte}(T, x)$ est le nombre d'occurrences de $x$ dans $T$ (si $x\notin T, \textsc{compte}(T, x) = 0$).
\end{questionpartie}
\begin{questionpartie}
    \label{naif}
    Utilisez la question précédente pour créer algorithme en $\mathcal{O}(n^2)$ pour résoudre le problème. Vous en donnerez le pseudo-code et la preuve.
\end{questionpartie}
\begin{questionpartie}
    Explicitez en quelques phrases le fonctionnement de l'algorithme de la question précédente si, en entrée, on lui donne les deux exemples du début de l'énoncé. 
\end{questionpartie}            
\begin{questionpartie}
    En déduire que la complexité du problème de l'élément majoritaire est en $\mathcal{O}(n^2)$
\end{questionpartie}
\end{partie}

\end{exo}


\clearpage
\begin{exo}[Tris]
    On va maintenant utiliser le tri pour résoudre le problème.
    
    \begin{partie}[Tris]
        \begin{questionpartie}
            Donnez le nom d'un algorithme de tri ayant une complexité de $\mathcal{O}(n\ln(n))$ avec $n$ la taille du tableau à trier. Vous expliciterez son fonctionnement en quelques phrases.
        \end{questionpartie}

        \begin{questionpartie}
            Justifiez en quelques phrases que le problème du tri est en $\Omega(n\ln(n))$.
        \end{questionpartie}

        \begin{questionpartie}
            Déduisez-en que le problème du tri est en $\Theta(n\ln(n))$.
        \end{questionpartie}
        \end{partie}
    
        \begin{partie}[Élément  majoritaire]
            On suppose que l'on possède un algorithme permettant de trier un tableau de $n$ entiers en $\mathcal{O}(n \log(n))$ opérations.
            \begin{questionpartie}
                Si $T$ est trié, démontrez l'existence d'un indice $i$ (que vous expliciterez) tel que, si $T$ admet un élément majoritaire $x$, alors $x=T[i]$. 
            \end{questionpartie}
            \begin{questionpartie}
                Déduisez-en un algorithme itératif, dont vous donnerez le pseudo-code et la preuve, plus efficace que celui de l'exercice~\ref{naif} pour résoudre le problème de l'élément majoritaire.            
            \end{questionpartie}
            \begin{questionpartie}
                Explicitez en quelques phrases le fonctionnement de l'algorithme de la question précédente si, en entrée, on lui donne les deux exemples du début de l'énoncé.
            \end{questionpartie}            
        \end{partie}   
    \end{exo}

\begin{exo}[Diviser pour régner]

\begin{partie}[Principe]

    Expliciter le principe algorithmique de {\it diviser pour régner}

\end{partie}
\begin{partie}[La taille du tableau est une puissance de 2]
    On suppose dans cet exercice que $n$ est une puissance de 2. On appelle $T_1$ le tableau constitué des $n/2$ premiers éléments de $T$ (en Python, $T_1 = T[0:n/2]$), et $T_2$ le tableau constitué des $n/2$ derniers (en Python, $T_2 = T[n/2 : n]$).
\begin{questionpartie}
Montrez que si $T$ admet un élément majoritaire $x$, alors $x$ est un élément majoritaire de $T_1$ ou $T_2$.
\end{questionpartie}
\begin{questionpartie}
    Donnez un algorithme qui détermine, quand $x$ est un élément majoritaire de $T_1$ (ou $T_2$), si $x$ est un élément majoritaire de $T$.
\end{questionpartie}
\begin{questionpartie}
    Déduisez-en un algorithme récursif, plus efficace que celui de l'exercice~\ref{naif}, pour résoudre le problème.
\end{questionpartie}
\end{partie}
\begin{partie}[La taille du tableau est quelconque]
    \begin{questionpartie}
        Reprenez cet exercice sans supposer que $n$ est une puissance de 2. Que faut-il changer pour que l'algorithme continue de fonctionner ?
    \end{questionpartie} 
    \begin{questionpartie}
        Explicitez en quelques phrases le fonctionnement de l'algorithme de la question précédente si, en entrée, on lui donne les deux exemples du début de l'énoncé.
    \end{questionpartie}     
\end{partie}
\end{exo}

\clearpage
\begin{exo}[Piles]
        Une pile est un type de données dont on peut créer un objet en $\mathcal{O}(1)$ opérations et possédant deux méthodes :
        \begin{itemize}
            \item \verb|empile(élément, pile)| qui ajoute un élément différent de \verb|None| à la structure en $\mathcal{O}(1)$ opérations (on considère que tenter d'empiler \verb|None| ne va pas provoquer d'erreurs et ne va rien ajouter à la structure).
            \item \verb|dépile(pile)| qui {\bf rend} et supprime de la structure le dernier élément à avoir été empilé en $\mathcal{O}(1)$ opérations. Si la pile est vide cette fonction va rendre \verb|None|
        \end{itemize}
        

        L'algorithme suivant va par exemple afficher à l'écran \verb|3|, \verb|1|, \verb|2|, \verb|4| puis \verb|None| :
        \begin{verbatim}
            crée une pile vide de nom P
            empile 4 dans P
            empile 3 dans P
            empile None dans P
            x = dépile(P)
            affiche à l'écran la variable x
            empile 2 dans P
            empile 1 dans P
            x = dépile(P)
            affiche à l'écran la variable x
            x = dépile(P)
            affiche à l'écran la variable x
            x = dépile(P)
            affiche à l'écran la variable x
            x = dépile(P)
            affiche à l'écran la variable x
        \end{verbatim}

        La pile est une structure permettant de garder en mémoire des choses à faire (en empilant) et de toujours effectuer la tâche la plus récente non réalisée (en dépilant).
        
    \begin{partie}[Structure de pile]

            Proposez une implémentation avec des listes de la structure de pile en python. Vous implémenterez les fonctions :
            \begin{itemize}
                \item \verb|crée_pile_vide() -> list|
                \item \verb|empile(x: int or None, pile: list|                
                \item \verb|dépile(pile:list) -> int or None|    
            \end{itemize}            
                
    
    \end{partie}
    \begin{partie}[Pile et élément majoritaire]
    
        On considère l'algorithme \textsc{Majorité} ci après.
            %\nocaptionofalgo
        \begin{algorithm}[h!]
            \DontPrintSemicolon
        \Entree{Un tableau d'entiers $T$}
        \Deb{
            Soient $P_1$ et $P_2$ deux piles vides.\;
            \PourCh{ $x$ de $T$}{
                $y= \textsl{dépile}(P_1)$ \;
                $\textsl{empile}(y, P_1)$ \;
                \eSi{y == None {\bf ou} y $\neq$ x}
                {
                    $\textsl{empile}(x, P_1)$ \;
                }
                {
                    $z=\textsl{dépile}(P_2)$ \;
                    \eSi {z == None {\bf ou} z == x}
                    {   
                        $\textsl{empile}(z, P_2)$ \;
                        $\textsl{empile}(x, P_2)$ \;                        
                    }
                    {
                        $\textsl{empile}(z, P_1)$ \;
                        $\textsl{empile}(x, P_1)$ \;
                    }
                }
            }
        }
        \caption{algorithme \textsc{Majorité}}
        \end{algorithm}

        \begin{questionpartie}
                Que vaut $y$ à la ligne 6 par rapport à $P_1$ ?
        \end{questionpartie}
        \begin{questionpartie}
            Montrez qu'à chaque itération de la boucle for, $x$ est empilé soit dans $P_1$, soit dans $P_2$.
        \end{questionpartie}
        \begin{questionpartie}
            Donnez le contenu des deux piles $P_1$ et $P_2$ {\bf à la fin de l'algorithme} si, en entrée, on lui donne les deux exemples du début de l'énoncé.
        \end{questionpartie}
        \begin{questionpartie}
            Quelle est la complexité de l'algorithme~?
        \end{questionpartie}
        \begin{questionpartie}
            Montrez que tous les éléments de la pile $P_2$ sont identiques à la fin de l'algorithme. 
        \end{questionpartie}
        \begin{questionpartie}
            Montrez qu'à la fin de l'algorithme, deux éléments consécutifs de la pile $P_1$ sont toujours différents.        
        \end{questionpartie}
        \begin{questionpartie}
        
            Déduisez-en qu'un élément majoritaire, s'il en existe, ne peut être que le dernier élément stocké dans $P_2$ ou le dernier élément stocké dans $P_1$.
        \end{questionpartie}
    \end{partie}
    \begin{partie}[Conclutions]        
        \begin{questionpartie}
            Déduisez de la partie précédente un algorithme linéaire en la taille du tableau en entrée pour résoudre le problème de l'élément majoritaire
        \end{questionpartie}             
        \begin{questionpartie}
            Quelle est la complexité du problème de l'élément majoritaire ?
        \end{questionpartie}             
    \end{partie}

\end{exo}
\end{document}
