% Options for packages loaded elsewhere
\PassOptionsToPackage{unicode}{hyperref}
\PassOptionsToPackage{hyphens}{url}
%
\documentclass[
]{article}
\usepackage{fullpage}
\usepackage{lmodern}
\usepackage{amssymb,amsmath}
\usepackage{ifxetex,ifluatex}
\ifnum 0\ifxetex 1\fi\ifluatex 1\fi=0 % if pdftex
  \usepackage[T1]{fontenc}
  \usepackage[utf8]{inputenc}
  \usepackage{textcomp} % provide euro and other symbols
\else % if luatex or xetex
  \usepackage{unicode-math}
  \defaultfontfeatures{Scale=MatchLowercase}
  \defaultfontfeatures[\rmfamily]{Ligatures=TeX,Scale=1}
\fi
% Use upquote if available, for straight quotes in verbatim environments
\IfFileExists{upquote.sty}{\usepackage{upquote}}{}
\IfFileExists{microtype.sty}{% use microtype if available
  \usepackage[]{microtype}
  \UseMicrotypeSet[protrusion]{basicmath} % disable protrusion for tt fonts
}{}
\makeatletter
\@ifundefined{KOMAClassName}{% if non-KOMA class
  \IfFileExists{parskip.sty}{%
    \usepackage{parskip}
  }{% else
    \setlength{\parindent}{0pt}
    \setlength{\parskip}{6pt plus 2pt minus 1pt}}
}{% if KOMA class
  \KOMAoptions{parskip=half}}
\makeatother
\usepackage{xcolor}
\IfFileExists{xurl.sty}{\usepackage{xurl}}{} % add URL line breaks if available
\IfFileExists{bookmark.sty}{\usepackage{bookmark}}{\usepackage{hyperref}}
\hypersetup{
  pdftitle={Algorithmes gloutons : comme solution exacte},
  hidelinks,
  pdfcreator={LaTeX via pandoc}}
\urlstyle{same} % disable monospaced font for URLs
\setlength{\emergencystretch}{3em} % prevent overfull lines
\providecommand{\tightlist}{%
  \setlength{\itemsep}{0pt}\setlength{\parskip}{0pt}}
\setcounter{secnumdepth}{-\maxdimen} % remove section numbering
\ifluatex
  \usepackage{selnolig}  % disable illegal ligatures
\fi

\title{Algorithmes gloutons : comme solution exacte}
\author{}
\date{}

\begin{document}
\maketitle

\hypertarget{but}{%
\subsection{But}\label{but}}

Montrer l'intérêt des algorithmes gloutons, la façon de les construire
et de prouver qu'ils fonctionnent. On s'attachera dans cette séance
tableau à prouver qu'ils rendent une solution optimale à un problème
donné.

\hypertarget{algortihme-glouton}{%
\subsection{Algortihme glouton}\label{algortihme-glouton}}

Un \href{https://fr.wikipedia.org/wiki/Algorithme_glouton}{algorithme
glouton} choisit à chaque étape la meilleure possibilité localement. Ce
type d'algorithmes est très utilisé pour résoudre des problèmes où l'on
veut une réponse rapide, mais pas forcément une réponse optimale. D'un
point de vue théorique, ces algorithmes sont extrêmement importants. Il
sont, par exemple, en bijection avec la
\href{https://fr.wikipedia.org/wiki/Matro\%C3\%AFde}{structure de
matroïde}.

Intérêt :

\begin{itemize}
\tightlist
\item
  donne toujours un résultat
\item
  souvent de complexité faible
\end{itemize}

Problème :

\begin{itemize}
\tightlist
\item
  ne donne pas forcément le meilleur résultat : une \emph{heuristique}
\item
  pas forcément de solution unique
\end{itemize}

Pour beaucoup de problèmes d'optimisation, un algorithme glouton est
optimal pour une version simplifiée du problème. Comme l'algorithme va
vite, on peut recommencer plusieurs fois pour trouver une meilleure
solution.

\hypertarget{optimalituxe9-et-glouton}{%
\subsubsection{optimalité et glouton}\label{optimalituxe9-et-glouton}}

Les problèmes d'optimalité demandent de trouver, parmi un ensemble de
solutions possible, une solution minimisant (ou maximisant) un critère.
Par exemple :

\begin{itemize}
\tightlist
\item
  pour un ensemble de coûts de constructions possibles d'une voiture,
  trouver celle qui minimise le coûts tout en maximisant la qualité
  totale des pièces,
\item
  parmi tous les parcours passant par un ensemble de villes donné,
  choisir celui qui minimise le nombre de kilomètres parcourus
\item
  maximiser le nombre de films projetés dans un multiplexe de cinéma
\item
  \ldots{}
\end{itemize}

La difficulté de ces problèmes vient du fait que l'on ne peut a priori
pas trouver la meilleure solution sans les examiner toutes. Et s'il y a
beaucoup de solutions ça peut prendre vraiment beaucoup de temps.

Certains problèmes cependant permettent d'être résolus en construisant
petit à petit une solution, sans jamais remettre en cause ses choix. On
peut alors souvent trouver très rapidement la meilleure solution
possible. On peut également utiliser cette solution construite petit à
petit pour trouver une solution approchée à un problème plus général.
Cette classe d'algorithmes qui construit itérativement d'une solution
est appelée \emph{algorithmes gloutons}.

\hypertarget{condition-nuxe9cessaire-et-suffisante-doptimalituxe9.}{%
\subsubsection{condition nécessaire et suffisante
d'optimalité.}\label{condition-nuxe9cessaire-et-suffisante-doptimalituxe9.}}

Pour qu'un algorithme glouton \textbf{trouve une solution optimale} il
faut :

\begin{itemize}
\tightlist
\item
  \textbf{initialisation} : montrer qu'il existe une solution optimale
  contenant le 1er choix de l'algorithme
\item
  \textbf{récurrence} : montrer que la première différence entre une
  solution optimale et la solution de l'algorithme ne peut résulter en
  une meilleure solution. Pour cela on cherchera une solution optimale
  dont les choix coïncident le plus longtemps possible avec la solution
  donnée par notre algorithme et on prouvera qu'elles coïncident jusqu'à
  la fin.
\end{itemize}

\hypertarget{exercice-1-le-gradient}{%
\subsection{exercice 1 : le gradient}\label{exercice-1-le-gradient}}

On suppose que l'on veuille trouver le minimum d'une fonction \(f\)
dérivable.

\begin{enumerate}
\def\labelenumi{\arabic{enumi}.}
\tightlist
\item
  Décrivez l'algorithme du gradient sous la forme d'un algorithme
  glouton
\item
  montrer qu'il peut converger vers la solution
\item
  montrer qu'il peut ne pas converger vers la solution
\end{enumerate}

\hypertarget{exercice-2-le-rendu-de-piuxe8ces}{%
\subsection{exercice 2 : le rendu de
pièces}\label{exercice-2-le-rendu-de-piuxe8ces}}

\hypertarget{un-systuxe8me-de-piuxe8ce-particulier}{%
\subsubsection{Un système de pièce
particulier}\label{un-systuxe8me-de-piuxe8ce-particulier}}

Proposez un algorithme glouton permettant de rendre la monnaie d'un
achat en un nombre minimum de pièces valant 5, 3 et 1 pokédollar.

Démontrez que votre algorithme est bien optimal.

\hypertarget{systuxe8me-de-piuxe8ces-quelconque}{%
\subsubsection{système de pièces quelconque
?}\label{systuxe8me-de-piuxe8ces-quelconque}}

\begin{itemize}
\tightlist
\item
  donnez une version générale de votre algorithme de rendu de pièces,
  c'est à dire où l'on a \(n\) pièces valant
  \(p_1 < p_2 < \dots < p_n\).
\item
  Cet algorithme glouton de va pas donner de solution optimale quelque
  soit le système de pièces, donnez un exemple pour lequel ça ne
  fonctionne pas.
\end{itemize}

\hypertarget{exercice-3-allocation-de-salles-de-cinuxe9ma}{%
\subsection{exercice 3 : allocation de salles de
cinéma}\label{exercice-3-allocation-de-salles-de-cinuxe9ma}}

Un gérant de cinéma a en sa possession \(m\) films caractérisés chacun
par des couples (\(d_i\), \(f_i\)) où \(d_i\) est l'heure de début du
film et \(f_i\) l'heure de fin. Il se pose 2 problèmes :

\begin{itemize}
\tightlist
\item
  Quel est le nombre maximum de films que je peux voir en une journée ?
\item
  Quel est le nombre minimum de salles à avoir pour visionner tous les
  films en stock.
\end{itemize}

\begin{quote}
\textbf{Nota Bene :} Si vous êtes d'humeur programmatrice, codez les
algorithmes de cet exercice. Si vous séchez, une solution possible vous
est donnée dans le corrigé.
\end{quote}

\hypertarget{voir-un-maximum-de-films}{%
\subsubsection{voir un maximum de
films}\label{voir-un-maximum-de-films}}

Proposez (et prouvez) un algorithme permettant de rendre une liste
maximale de films à voir en une journée.

\hypertarget{nombre-minimum-de-salles-pour-placer-tous-les-films-en-stock}{%
\subsubsection{nombre minimum de salles pour placer tous les films en
stock}\label{nombre-minimum-de-salles-pour-placer-tous-les-films-en-stock}}

Proposez (et prouvez) un algorithme permettant de rendre le nombre
minimum de salles et son organisation permettant de projeter tous les
films.

\hypertarget{exercice-4-ordonnancement}{%
\subsection{exercice 4 :
ordonnancement}\label{exercice-4-ordonnancement}}

Les problèmes d'ordonnancement sont multiples. Certains sont durs
d'autres faciles. Mais un algorithme glouton permet de trouver souvent
une solution acceptable pour beaucoup d'entres eux et même parfois
optimale pour certains problèmes.

Le problème suivant est résoluble par un algorithme glouton : On
considère \(m\) produits de durée 1 à fabriquer. Si le produit \(i\) est
réalisée avant la date \(d_i\) on peut le vendre pour un prix \(p_i\),
sinon il est invendable. Proposez un algorithme permettant de maximiser
les profits en considérant que l'on a qu'un seul ouvrier.

\hypertarget{ensemble-compatible}{%
\subsubsection{ensemble compatible}\label{ensemble-compatible}}

Un ensemble de produits est dit \emph{compatible} s'il existe un
ordonnancement de leur production permettant de tous les vendre (chaque
produit est fabriqué avant sa date de péremption).

Montrer qu'un ensemble de produits est compatible si et seulement si
l'ordonnancement par date \(d_i\) croissante permet de tous les vendre.

\hypertarget{algorithme}{%
\subsubsection{algorithme}\label{algorithme}}

Montrer que l'algorithme glouton suivant est optimal :

\begin{enumerate}
\def\labelenumi{\arabic{enumi}.}
\tightlist
\item
  on trie les produits par prix décroissant
\item
  ensemble = \{\}
\item
  pour chaque produit x dans cet ordre : on ajoute x à ensemble s'il
  reste compatible
\item
  rendre ensemble (qui est un ensemble de profit maximal)
\end{enumerate}

\begin{quote}
\textbf{Nota bene :} pour la preuve, on pourra comparer une solution
optimale et la solution donnée par notre algorithme en regardant la
première différence.
\end{quote}

\end{document}
