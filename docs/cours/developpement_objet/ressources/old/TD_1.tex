\documentclass[12pt]{article}
\usepackage[utf8]{inputenc}

\usepackage[pdftex,a4paper,bookmarks]{hyperref}
\usepackage{fullpage}


\usepackage{url}
\usepackage{listings}
\usepackage{courier}
\lstset{basicstyle=\ttfamily}

\begin{document}


    \begin{center}
      \begin{tabular}{c}
      \hline
    \\
        {\bf \textsf {\Large Programmation Orientée Objet}}\\
    \\
        {\bf \textsf {\Large TD 1 : Classes et objets}}\\
    \\
        \hline
      \end{tabular}
    \end{center}
    \vspace{15mm}

\section{Un Dé}

\subsection{Modèle d'un dé classique}

On veut créer une classe \verb|Dice|. Elle doit être capable de :
\begin{itemize}
	\item créer un objet sans paramètre,
	\item créer un objet avec sa valeur initiale,
	\item connaître et donner la valeur du dé (avec les méthodes \verb|get_position| et \verb|set_position|),
	\item lancer un dé,
	\item connaître le nombre maximum de faces (\verb|NUMBER_FACES|), commun à tous les dés.
\end{itemize}

Proposez une modélisation UML de la classe \verb|Dice| et écrivez des tests (en python) permettant de vérifier que tout se passe comme prévu.

\subsection{Modèle d'un dé pipé}

On souhaite améliorer notre classe \verb|Dice| pour pouvoir tricher. Complétez la modélisation UML précédente afin d'associer à chaque dé une liste correspondant aux probabilités de tomber sur chaque face.

Que faut-il modifier ?


\subsection{Namespaces}

Exécutez le code suivant en montrant toutes les lignes de codes et les {\em Namespaces} utilisés.

\lstset{language=Python}
\begin{lstlisting}
from dice import Dice

dice = Dice()
print(dice.position)
dice.roll()
print(dice.get_position())
print(dice.NUMBER_FACES)
\end{lstlisting}

\section{Des Dés}

Pour pouvoir jouer à des jeux de dés, implémentons une classe \verb|TapisVert|. Cette classe doit avoir~:
\begin{itemize}
	\item 5 dés comme attribut,
	\item pouvoir lancer les 5 dés simultanément ou individuellement,
	\item Connaître la valeur d'un dé spécifique.
\end{itemize}

\end{document}


%%% Local Variables:
%%% mode: latex
%%% TeX-master: t
%%% End:
