\documentclass[12pt]{article}
    \usepackage[utf8]{inputenc}
    
    \usepackage[pdftex,a4paper,bookmarks]{hyperref}
    \usepackage{fullpage}
    
    
    \usepackage{url}
    \usepackage{listings}
    \usepackage{courier}
    \lstset{basicstyle=\ttfamily}
    
    
    \begin{document}
    
    
        \begin{center}
          \begin{tabular}{c}
          \hline
        \\
            {\bf \textsf {\Large Programmation Orientée Objet}}\\
        \\
            {\bf \textsf {\Large TD 4 : Programmation évènementielle \& UI}}\\
        \\
            \hline
          \end{tabular}
        \end{center}
        \vspace{15mm}
   
    
    \section{Programmation d'UI}
    
    On veut créer une fenêtre contenant un champ texte de valeur initiale 0 et un bouton. Lorsque l'on clique sur le bouton, la valeur du texte augmente de 1.
    Proposez un design de fenêtre permettant de réaliser cette interface.
    Comment la testeriez-vous~?
    
    \subsection{Modèle MVC}
    
    Un façon pratique de développer une interface est d'utiliser le modèle MVC (Modèle, Vue, Contrôleur) :
    \begin{itemize}
        \item le {\bf Modèle} régit l'accès aux données
      \item la {\bf Vue} est tout ce qui est affiché
      \item {\bf Contrôleur} est ce qui fait le lien entre Vue et Modèle
    
    \end{itemize}
    Proposez un modèle UML en 3 classes (Modèle, Vue et Contrôleur) permettant de réaliser cette application.Comment la testeriez-vous avec des tests unitaires ?
    
    \subsubsection{MVC et appjar.info}
    
    Le code suivant utilise la bibliothèque \url{appjar.info} pour coder l'UI. Associez le code ci-dessus au modèle MVC. Où sont les différentes parties ? 
    
    \lstset{language=Python}
    \begin{lstlisting}
    from appJar import gui
    
    model = {
        "value": 0
    }
    
    def press(button):
        model["value"] += 1
        app.setLabel("value", str(model["value"]))
    
    app = gui()
    app.addLabel("value", str(model["value"]), 0, 0)
    app.addButton("+1", press, 0, 1)
    
    app.go()
    print("c'est fini.")
    \end{lstlisting}
    
    
    
    
    \subsubsection{Ajout de boutons}
    
    Ajoutez à  l'application un bouton `-1' qui décrémente la valeur affichée de 1 si cette valeur est $>0$.


\section{Fonctions et namespaces}

\subsection{Les fonctions sont des variables comme les autres}


\lstset{language=Python}
\begin{lstlisting}
une_liste = []
une_liste.append("premier")
truc = une_liste.append
truc("?")

print(une_liste)
\end{lstlisting}


Dans le code ci-dessus, que représente \verb|append|~? Exécutez-le en montrant toutes les lignes de codes et les {\em namespaces} utilisés.


\subsection{Fonctions de fonctions}

On souhaite créer une fonction \verb|ajoute| avec un paramètre (entier) \verb|x|. Le retour de cette fonction doit être une fonction à un paramètre \verb|y| qui rend \verb|x + y|

\subsubsection{Implémentation}

Codez cette fonction et vérifiez (à la main) qu'elle passe bien vos tests en notant toutes les variables et les namespaces rencontrés.


    
    \section{Les dés}
    
    Proposez l'UI et le modèle UML d'une fenêtre permettant de lancer un dé. Comment coderiez vous le tout avec \url{appjar.info} ?
    
    \end{document}
    
    
    %%% Local Variables:
    %%% mode: latex
    %%% TeX-master: t
    %%% End:
    