\documentclass
[12pt]
{article}
\usepackage[utf8]{inputenc}

\usepackage{amssymb}
\usepackage[ruled, vlined, french]{algorithm2e}
\usepackage[pdftex]{graphicx}
\usepackage{hyperref}

\usepackage{fullpage} \usepackage{setspace}

\begin{document}

\begin{center}
  \begin{tabular}{c}
  \hline\\%\vspace{0.1cm}
  {\textsc{mpc\Large{I}}}\vspace{0.1cm}
  \\
%  
    {\bf \Large \sc Programmation et Algorithmes} \\ 
    \vspace*{.2cm}
    {\bf Examen Terminal} \\
    {\footnotesize Vendredi 13 mai 2022}\\
    \hline
  \end{tabular}
\end{center}
\vspace{0.6cm}
%
%
\begin{center}
{\em On rappelle qu'aucun document, ni équipement électrique ou électronique n'est autorisé. }

~\\

{\em Lorsque l'on vous demande d'écrire de décrire ou de donner un algorithme cela signifiera toujours en donner un pseudo-code}

\end{center}

\section{Chaines de caractères}

On rappelle qu'une chaine de caractères est une suite finie $a = a_0\dots a_{n-1}$ d'un alphabet fini $\mathcal{A}$. Une suite $b=b_0\dots b_{m-1}$ est une {\em sous-suite} de $a$ s'il existe $0 \leq i < n$ tel que $a_{i + j} = b_j$ pour tout $j$ allant de $0$ à $m-1$.

\subsection{Algorithme naïf\label{naif}}

\begin{enumerate}
    \item donnez l'algorithme {\em naïf} (c'est à dire qu'il reproduit exactement la définition) permettant de savoir si $b$ est une sous-chaine de $a$ ou non.
    \item donnez et justifiez la complexité maximale et minimale de cet algorithme.
    \item rappelez et donnez un argument de justification concernant sa complexité en moyenne lorsque $b$ n'est pas une sous-chaine de $a$.
\end{enumerate}

\subsection{Nombre d'occurrences\label{naif2}}

\begin{enumerate}
    \item modifiez l'algorithme de la question~\ref{naif} pour qu'il rende le nombre de fois où $b$ est une sous-chaine de $a$. Vous justifierez votre algorithme.
    \item donnez et justifiez la complexité maximale et minimale de cet algorithme.
\end{enumerate}

\subsection{Modification}

L'algorithme de Boyer-Moore-Horspool est une modification de l'algorithme de recherche naïf. Il fonctionne de la façon suivante :

\begin{itemize}
    \item pour un indice $i$ donné $j$ décroit de $m-1$ à $0$.
    \item lors de la première non correspondance ($a_{i + j} \neq b_j$ et $a_{i + j'} = b_{j'}$ pour $j < j' < m$) $i$ est décalé de :
    \begin{itemize}
        \item $m$ si $a_{i +m-1}$ n'est pas un caractère de $b$.
        \item $m-k-1$ où $k$ est l'indice le plus grand dans $b$ du caractère $a_{i + m -1}$.
    \end{itemize}
\end{itemize}

\begin{enumerate}
    \item démontrez que ce fonctionnement est bien une façon valide de chercher si $b$ est une sous-chaine de $a$.
    \item proposez une structure permettant de connaitre le décalage à effectuer en $\mathcal{O}(1)$ opérations en moyenne ({\em indication} : vous pourrez utiliser un dictionnaire), puis utilisez la pour écrire un algorithme permettant de calculer le décalage. Vous en donnerez la complexité.
    \item proposez une modification de l'algorithme de la question~\ref{naif2} qui utilise la méthode de Boyer-Moore-Horspool pour la recherche.
    \item donnez et justifiez la complexité maximale et minimale de cet algorithme.
\end{enumerate}

\section{Algorithme glouton}

On appelle {\em équilibrage de charge} le problème suivant : 

\begin{itemize}
    \item on possède $m$ machines et $n$ tâches à effectuer.
    \item chaque tâche $j$ nécessite $t_j$ unités de temps pour être effectuée par une machine.
    \item pour chaque machine $i$, on associe l'ensemble $M_i$ des tâches effectuées par celles-ci, et on note $T_i$ le temps passé à effectuer ses tâches : $T_i = \sum_{j \in M_i} t_j$.
\end{itemize}

On cherche à trouver les ensembles $M_i$ permettant de minimiser la quantité : $\max_{1\leq i \leq m} T_i$. On note $T^\star$ ce minimum.

\subsection{Quelques propriétés}
\begin{enumerate}
    \item montrez que l'on a $T^\star \geq \max_{1 \leq j\leq n} t_j$.
    \item montrez que l'on a $T^\star \geq \frac{1}{m}\sum_{1 \leq j\leq n} t_j$.
\end{enumerate}

\subsection{Un algorithme glouton}

\begin{enumerate}
    \item Proposez un algorithme glouton permettant de résoudre le problème. Cet algorithme glouton ajoutera itérativement une tâche à la machine $i$ réalisant $T_i = \min_j T_j$.
    \item dans quel ordre proposez vous de ranger les tâches ? Justifiez votre réponse.
    \item montrez que s'il y a $m$ tâches ou moins à classer, l'algorithme glouton trouve la solution optimale.
\end{enumerate}

\subsection{Propriétés}

On considère une réalisation de l'algorithme. Soit $i^\star$ la machine réalisant $T_{i^\star} = \max_{1\leq i \leq m} T_i$ à la fin de l'algorithme, et $j$ l'indice de la dernière tâche qui lui a été assignée au cours de l'exécution de l'algorithme.

\begin{enumerate}
    \item montrez qu'à la fin de l'algorithme, on a $T_{i^\star} -t_j \leq T_k$ pour tout $k$.
    \item en déduire que $T_{i ^\star} - t_j \leq \frac{1}{m}\sum_{1\leq k\leq m}T_k$.
    \item déduire de la déduction que $T_{i ^\star} - t_j \leq T^\star$.
    \item puis que $T_{i ^\star} \leq 2 \cdot T^\star$.
\end{enumerate}

\subsection{Performance}

\begin{enumerate}
    \item en utilisant 2.3.4, montrez que la solution proposée par l'algorithme glouton est au pire 2 fois moins bonne que la solution optimale.
    \item montrer que cette performance est atteinte quelque soit l'ordre des tâches utilisé.
\end{enumerate}

\section{Problème}

Ce problème est une adaptation d'un problème donné au phases de qualification de la coding jam de google 2022 (\url{https://codingcompetitions.withgoogle.com/codejam}) qui est un concours d'algorithmie et de code (comme quoi l'algorithmie sert pour trouver du travail :-)). Il n'y avait bien sur aucune indication pour résoudre le problème.

\begin{itemize}
    \item Une chaine de caractères $a=a_0\dots a_{n-1}$ est appelée une {\em tour} si pour tous $0\leq i < j < n$ tels que $a_i = a_j$, alors $a_i = a_k$ quelque soit $i \leq k \leq j$.
    \item la concaténation $+$ de 2 chaines $a=a_0\dots a_{n-1}$ et $b = b_0\dots b_{m-1}$ est la chaine $a+b = a_0\dots a_{n-1}b_0 \dots b_{m-1}$.
    \item pour une chaine $a = a_0\dots a_{n-1}$ donnée, on notera $a_{-1} = a_{n-1}$.
\end{itemize}

Le problème est le suivant : on possède un ensemble $\mathcal{T}$ de $k$ tours, peut-on concaténer ces $k$ tours entre elles pour former une tour ?

Par exemple :

\begin{itemize}
    \item $aaaaabbbbbc$ est une tour.
    \item $abba$ n'est pas une tour.
    \item si $\mathcal{T} = \{ aaaaabbbbbc, ddd, cc \}$, il existe une solution : $aaaaabbbbbc + cc + ddd = aaaaabbbbbcccddd$. Notez que toutes les concaténations ne fonctionnent pas forcément, la concaténation $cc + aaaaabbbbbc + ddd = ccaaaaabbbbbcddd$ ne donne pas une tour par exemple.
    \item si $\mathcal{T} = \{ aaaaabbbbbc, ddd, b \}$, il n'existe pas de solution.
\end{itemize}

\subsection{Vérification}

Soit $a$ une chaine de caractères. Ecrivez un algorithme qui vérifie si $a$ est une tour ou non.

Vous prouverez la véracité et vous indiquerez la complexité de cet algorithme.

\subsection{Réduction du problème}

\begin{enumerate}
    \item montrer que s'il existe une solution au problème, alors pour toute tour $a \in \mathcal{T}$ telle que $a_0 = a_{-1}$, s'il existe $b \in \mathcal{T}$ telle que $b \neq a$ et $b_0 = a_0$  (respectivement $b_{-1} = a_0$), alors il existe une solution à $\mathcal{T} \backslash \{a, b\} \cup \{ c \}$ avec $c = a + b$ (respectivement $c=b+a$) et cette solution est aussi solution pour $\mathcal{T}$.
    \item montrer que s'il existe une solution au problème, alors pour toute tour $a \in \mathcal{T}$ telle que $a_0 \neq a_{-1}$ il existe au plus une tour $b \neq a$ avec $b_0 \neq b_{-1}$ telle que $a_0 = b_{-1}$ et au plus une tour $b \neq a$ avec $b_0 \neq b_{-1}$ telle que $a_{-1} = b_0$. De plus, si ces deux tours existent, elles sont différentes.
    \item montrer que s'il existe une solution au problème, alors pour toute paire de tours $a \neq b \in \mathcal{T}$ telles que $a_{-1} = b_0$ (respectivement $a_0 = b_{-1}$), alors il existe une solution à $\mathcal{T} \backslash \{a, b\} \cup \{ c \}$ avec $c = a + b$ (respectivement $c=b+a$)  et cette solution est aussi solution pour $\mathcal{T}$.
    \item montrer que s'il existe une solution au problème, alors s'il existe une tour $a \in \mathcal{T}$ telle qu'il n'existe pas une autre tour $b \in \mathcal{T}$ avec $a_0 = b_{-1}$ ou $a_{-1} = b_0$ alors :
    \begin{itemize}
        \item il existe une solution $t^\star$ à $\mathcal{T}^\star = \mathcal{T} \backslash \{ a \}$.
        \item $a + t^\star$ est une solution de $\mathcal{T}$.
    \end{itemize}
    \end{enumerate}

\subsection{Résolution du problème}

En utilisant la partie précédente, donnez un algorithme pour résoudre le problème.
\end{document}


%%% Local Variables: 
%%% mode: latex
%%% TeX-master: t
%%% End: 
