\documentclass[12pt]{article}
\usepackage[utf8]{inputenc}

\usepackage[pdftex,a4paper,bookmarks]{hyperref}
\usepackage{fullpage}


\usepackage{url}
\usepackage{listings}
\usepackage{courier}
\lstset{basicstyle=\ttfamily}

\begin{document}


    \begin{center}
      \begin{tabular}{c}
      \hline
    \\
        {\bf \textsf {\Large Programmation Orientée Objet}}\\
    \\
        {\bf \textsf {\Large TD 1 : Classes et objets}}\\
    \\
        \hline
      \end{tabular}
    \end{center}
    \vspace{15mm}

\section{Compteur}
\subsection{Compteur simple\label{execution}}
On souhaite créer un objet \verb|Compteur| qui retient le compte de quelque chose et est capable d'ajouter 1 à son
compte quand on le lui demande.
Proposez une modélisation UML de cet objet simple. Cette modélisation doit être capable de répondre au code suivant~: \\


\begin{lstlisting}
from counter import Counter
    
c1 = Counter()
c2 = Counter()
c1.count()
c2.count()
c1.count()

print(c2.get_value())
\end{lstlisting}
Que valent \verb|c1.get_value()|, \verb|c2.get_value()| à la fin de l'exécution ? 

\subsection{Code de classes}
Le fichier \verb|counter.py| contient le code la classe \verb|Counter|~:

\begin{lstlisting}
class Counter:
    def __init__(self):
        self.value = 0
           
    def count(self):
        self.value = self.value + 1
    
    def get_value(self):
        return self.value
\end{lstlisting}

Exécutez le code de la partie~\ref{execution} un utilisant le code python de la classe ci-avant. Montrez tous les {\em namespaces} utilisés.


\subsection{Compteur à pas choisi}
On souhaite maintenant pouvoir choisir le pas de notre compteur (c'est-à-dire ajouter 2 à chaque fois plutôt que 1 par
exemple). Que faut-il ajouter à notre classe ?


\section{Un Dé}

\subsection{Modèle d'un dé classique}

On veut créer une classe \verb|Dice|. Elle doit être capable de :
\begin{itemize}
	\item créer un objet sans paramètre,
	\item créer un objet avec sa valeur initiale,
	\item connaître et donner la valeur du dé (avec les méthodes \verb|get_position| et \verb|set_position|),
	\item lancer un dé.
\end{itemize}

Proposez une modélisation UML de la classe \verb|Dice|. Donnez des exemples de manipulation d'objets de cette classe :
créer un objet, modifier la valeur de sa position, obtenir sa position et le lancer.

\subsection{Modèle d'un dé pipé}

On souhaite améliorer notre classe \verb|Dice| pour pouvoir tricher. Complétez la modélisation UML précédente afin
d'associer à chaque dé une liste correspondant aux probabilités de tomber sur chaque face.

Que faut-il modifier ?

\end{document}


%%% Local Variables:
%%% mode: latex
%%% TeX-master: t
%%% End:
