\documentclass[12pt]{article}
\usepackage[utf8]{inputenc}

\usepackage[pdftex,a4paper,bookmarks]{hyperref}
\usepackage{fullpage}


\usepackage{url}
\usepackage{listings}
\usepackage{courier}
\lstset{basicstyle=\ttfamily}


\begin{document}


    \begin{center}
      \begin{tabular}{c}
      \hline
    \\
        {\bf \textsf {\Large Programmation Orientée Objet}}\\
    \\
        {\bf \textsf {\Large TD 2 : Composition et agrégation}}\\
    \\
        \hline
      \end{tabular}
    \end{center}
    \vspace{15mm}

\section{Des Dés}

Nous allons réutiliser les dés de la séance 1 pour les composer à une autre classe qui permettra de jouer à un jeu comportant plusieurs dés.

\subsection{Un dé}

Rappelez le modèle UML du dé de la séance 1 et donnez un petit exemple de programme lançant 5 dés différents.


\subsection{Tapis Vert}

Pour pouvoir jouer à des jeux de dés, implémentons une classe
\verb|TapisVert|. Cette classe doit avoir~:
\begin{itemize}
	\item 5 dés comme attribut (une liste de 5 dés nommée \verb|dices|),
	\item pouvoir lancer les 5 dés simultanément (méthode \verb|roll|),
	\item Connaître la somme des valeurs des dés (méthode \verb|sum|).
\end{itemize}

\begin{enumerate}
    \item retranscrivez le code la question précédente en utilisant ma classe \verb|TapisVert|
    \item Proposez un modèle UML pour cette classe. Quel lien la classe \verb|TapisVert| a-t-elle avec la classe \verb|Dice| ?
    \item Ecrire le code python de la classe.
    \item comment est-il possible d'avoir à la fois une méthode \verb|roll| pour \verb|Dice| et pour \verb|TapisVert| sans que python s'embrouille ?
    \item exécutez le code précédent en explicitant tous les namespaces utilisées (namespace de classe, d'objet, de fichier et de fonctions)
\end{enumerate}


\section{Des Cartes}
\subsection{Une carte}
Donnez le diagramme UML d'une classe \verb|Card| définie par une couleur et une valeur. 

\subsection{Un tas de cartes}
Un \verb|Deck| est un tas de cartes, initialement vide auquel on peut ajouter une carte, dont on peut voir la carte du
dessus et dont on peut prendre la carte du dessus (la piocher i.e la récupérer et l'enlever du paquet). Proposez un
diagramme UML pour cette classe. Quel est son lien avec la classe \verb|Card| ?

\subsection{Les couleurs}
On veut que les couleurs des cartes soient communes à toutes les cartes. Une façon de faire est de mettre les différentes possibilités {\em dans}  la classe carte. Proposez une modélisation UML/python de ceci. 


\end{document}

%%% Local Variables:
%%% mode: latex
%%% TeX-master: t
%%% End:
